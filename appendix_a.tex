
\section{Deriving the ELBO}
\label{append_a}
\subsection{The problem}

Given our generative model and our data, we would
like to find a posterior distribution: $P(Z \mid  X)$. Using
Bayes’ Rule, we get:
$P(Z\mid X) = \frac{P(X\mid Z)P(Z)}{P(X)} = \frac{P(X\mid Z)P(Z)}{\sum\limits_{\forall Z} P(X\mid Z) P(Z)}$
We call the numerator of the fraction on the right the “generative
model”. It is composed of the product of the likelihood *of the
hypothesis* ($P(X\mid Z)$) and the prior probability of the hypothesis
($P(Z)$). Note that the former is not a probability but a measure of how
well our hypothesis fits the data. The denominator is the ``marginal
likelihood'' of the data, $P(X)$. To find this, we need to marginalize
out (sum over) all possible hypotheses. Because the hypotheses are the
range of values for all of the latent variables in our model, this
summation is computationally intractable. In order to obtain a
posterior, we are forced to use some approximate means of finding this
denominator. Often, a sampling approach is used. However, sampling can
be very slow to converge and is not easily parallelizable across
multiple cores. The variational Bayesian approach, on the other hand,
treats the problem of finding an appropriate marginal distribution as an
optimization problem.

 \begin{itemize}
\item   Let $Z$ be our set of hidden variable collections: 
\item   Let $\Phi$ be the collection of all model parameters (Pitman-Yor
    parameters $a,b$ and Dirichlet distribution parameter $\alpha$.
\item   Let $X$ be the set of observations. In the case of word
    segmentation, for example, these would be each string of unsegmented
    phonemes.
\item   Note that our goal is to find $P(Z\mid X)$, the posterior (where $Z$ is
    the set of latent variables)
\item  recall $P(Z\mid X) = \frac{P(X\mid Z, \Phi)P(Z \mid  \Phi)}{\sum\limits_{\forall Z} P(X\mid Z, \Phi) P(Z\mid \Phi) } $ 
\end{itemize}

\subsection{Important formulae}


\subsubsection{Jensen's inequality}
Jensen’s inequality states that for a convex function $f $ and random
variable $X$: 
\begin{align}
f(\mathbb{E}[X]) \leq \mathbb{E}[f(X)] 
\end{align}
We are using the logarithm of the probability, so the function is actually concave. Jensen's inequality works both ways, meaning we switch the direction of the inequality:
    \begin{align}\log(\mathbb{E}[X]) \geq \mathbb{E}[\log(X)] \end{align}

\subsubsection{Expected value}
Note that for discrete random variables (which we are dealing with in this case)
\begin{align} \mathbb{E}_q(f(x)) = \sum\limits_{\forall x} q(x)f(x) \end{align}

\subsubsection{Logarithms}

Throughout this derivation (and the variational literature as a whole) the logarithm of the probability is used. There are various reasons to do this. Firstly, taking the logarithm allows us to use information-theoretic measures such as entropy. Furthermore, logarithms allow us to transform expensive multiplication and division into cheaper addition and subtraction, and helps when working with probabilities below the floating-point precision bound. Recall some facts about logarithms:

\begin{itemize}

\item $\lim\limits_{n\rightarrow 0} \log\ n = -\infty $
\item $\log\ AB = \log\ A + \log\ B $
\item $\log\ \frac{A}{B} = \log\ A - \log\ B $
\end{itemize}


\subsection{Derivation of variational bound}

The value we are looking to approximate is our denominator, which is the
likelihood of the data integrated over all hypotheses. Recall that our
inference problem lies in finding the denominator to the Bayesian
equation
$P(Z\mid X) = \frac{P(X\mid Z)P(Z)}{\int\limits_{\forall Z} P(X\mid Z) P(Z) dZ}$
Our hypotheses in this case are possible values for the latent variables
in the model. This integral (or in the discrete case, summation) is often
computationally intractable, so we use a variational approximation for
it.\
One way we can do this is by using the Kullback Leibler (KL) divergence between this
intractable integral and some variational distribution $q$.\

\begin{enumerate}
\item Let $q_\nu(Z)$ be a family of variational distributions with variational parameter $\nu$.
\item to get the marginal likelihood ($\log\ p(X\mid \Phi)$) we will take the KL divergence between $q_\nu(Z)$ and $p(Z\mid X,\Phi)$.
\item KL divergence is given by:
\begin{align}
\nonumber D_{KL}(q_\nu (Z) \mid \mid  p(Z\mid X,\Phi)) = \mathbb{E}_q[\log\ \frac{q_\nu(Z)}{p(Z\mid X,\Phi)}] \\
\nonumber  = \mathbb{E}_q [\log\ q_\nu(Z)- \log\ p(Z\mid X, \Phi)] \\
\nonumber  = \mathbb{E}_q [\log\ q_\nu(Z)- \log\ \frac{p(Z,X\mid \Phi)}{p(X\mid \Phi)}] \\
\nonumber  = \mathbb{E}_q [\log\ q_\nu(Z)- (\log\ p(Z,X\mid \Phi) - \log\ p(X\mid \Phi))] \\
 = \mathbb{E}_q [\log\ q_\nu(Z)] - \mathbb{E}_q [\log\ p(Z,X\mid \Phi)] + \log\ p(X\mid \Phi) 
\end{align}
\citep{blei:2006} 

\end{enumerate}
If we think about what KL divergence represents, we can intuitively understand why it cannot be negative. From here, we can see how minimizing this equation is the same as maximizing the lower bound on $\log\ p(X\mid \Phi)$:


\begin{align}
\nonumber 0 \leq \mathbb{E}_q [\log\ q_\nu(Z)] - \mathbb{E}_q [\log\ p(Z,X\mid \Phi)] + \log\ p(X\mid \Phi)\\
\nonumber - \log\ p(X\mid \Phi) \leq \mathbb{E}_q [\log\ q_\nu(Z)] - \mathbb{E}_q [\log\ p(Z,X\mid \Phi)]  \\
\log\ p(X\mid \Phi) \geq \mathbb{E}_q [\log\ p(Z,X\mid \Phi)] - \mathbb{E}_q [\log\ q_\nu(Z)] 
\end{align}


Another way to reach this same equation is by using Jensen's inequality. Consider the log marginal likelihood:

\begin{align}\log\ p(X\mid \Phi) = \log\ \sum\limits_{z \in \mathbf{Z}} p(X,z\mid \Phi)\end{align}
The sum marginalizes out the hidden variables $z$ in the joint probability distribution. Picking any variational distribution $q(z)$ we can multiply them by $\frac{q(z)}{q(z)}$:


\begin{align} \log\ \sum\limits_{\forall z \in \mathbf{Z}} ( p(x,z\mid \Phi) * \frac{q(z)}{q(z)} ) = \log\ \sum\limits_{\forall z \in \mathbf{Z}} q(z) \frac{ p(x,z\mid \Phi) }{q(z)}\end{align}
Jensen's inequality implies
\begin{align}\log\ \sum\limits_{\forall z \in \mathbf{Z}} q(z) \frac{p(x,z\mid \Phi) }{q(z)}  \geq \sum\limits_{\forall z \in \mathbf{Z}} q(z) \log\ \frac{ p(x,z\mid \Phi) }{q(z)} \end{align}
This equation can be broken into: 

\begin{align}
\nonumber \sum\limits_{\forall z \in \mathbf{Z}} q(z) \log\ \frac{ p(x,z\mid \Phi) }{q(z)}= \sum\limits_{\forall z \in \mathbf{Z}} q(z) (\log p(x,z\mid \Phi) - \log q(z)) = \\
 \nonumber \sum\limits_{\forall z \in \mathbf{Z}} q(z) \log p(x,z\mid \Phi) - \sum\limits_{\forall z \in \mathbf{Z}}  q(z)\log\ q(z) =  \\
 \sum\limits_{\forall z \in \mathbf{Z}} q(z) \log p(x,z\mid \Phi) + \mathcal{H}(q)\\
\end{align}

where 

\begin{align}
\mathcal{H}(q) =  - \sum\limits_{\forall z \in \mathbf{Z}}  q(z)\log\ q(z) \
\end{align} \citep{blei:2017} This first term is of the form of our expected value definition, so our equation becomes:


\begin{align} \log\ p(x\mid \Phi) \geq \mathbb{E}_q[\log\ p(x,z\mid \Phi)] + \mathcal{H}(q) \end{align}


This derivation yields an important fact: 

\begin{align}
\log\ p(X\mid \Phi) - KL(q(Z) \mid \mid  p(Z\mid X, \Phi)) = \mathbb{E}_q[\log\ p(z,x \mid  \Phi)] + H(q) \end{align}

From this equation, we can see why minimizing KL divergence gives us the best possible value for our marginal likelihood.



